% !TeX spellcheck = en_US
%\documentclass[11pt,a4paper]{article}
\documentclass[11pt
  , a4paper
  , article
  , oneside
%  , twoside
%  , draft
]{memoir}

\usepackage{control}
\usepackage[numbers]{natbib}


\begin{document}

\newcommand{\technumber}{
  RAON Control-Document Series\\
  Revision : v1.0,   Release : 2015-03-03 fixed date}
\title{\textbf{Timing System 개발환경 구성}}

\author{이상일\thanks{silee7103@ibs.re.kr} \\

  Rare Isotope Science Project\\
  Institute for Basic Science, Daejeon, South Korea
}
\date{\today}

\renewcommand{\maketitlehooka}{\begin{flushright}\textsf{\technumber}\end{flushright}}
%\renewcommand{\maketitlehookb}{\centering\textsf{\subtitle}}
%\renewcommand{\maketitlehookc}{C}
%\renewcommand{\maketitlehookd}{D}

\maketitle

\begin{abstract}
RAON accelerator는 대형 실험 장치로 많은 실험 장치들과 부대시설 장치로 구성되어 운영된다. 이러한 많은 실험 장치들을 제어하기 위한 제어 시스템들은 넓은 범위로 분산되어 구성되어 있으며 이러한 분산 환경에서의 많은 제어시스템들을 전체의 하나의 제어시스템으로 운영하기 위하여 RAON control system은 정밀한 타임 동기화가 필요하다. RAON 제어시스템은 정밀한 타이밍 동기화를 위하여 EVG/EVR 시스템을 사용한다. 본 문서는 EPICS와 연동되어 사용 되는 EVG/EVR Timing 시스템 개발 및 운영을 위한 환경 구성에 대하여 설명한다.
\end{abstract}

RAON accelerator에서 사용되는 Timing System은 Micro-Research Finland Oy 사의 EVG(Event Gengerator)/EVR(Event Receiver) 시스템을 사용한다. EVG/EVR Timing System은 많은 대형장치에서 사용되고 있는 검증된 Timing System 이다. EVG/EVR Timing System에 대한 특징은 아래와 같다.

\begin{itemize}
	\item Event driven system, 255 event codes
	\item 외부 RF reference clock을 이용한 event 신호 생성
	\item 50 ~ 125MHz Event clock rate
	\item Events generated
	\begin{itemize}
			\item From external HW inputs
			\item Two sequencers (up to 2048 events/sequencer)
			\item Multi counters
	\end{itemize}
	\item Cascaded Event Generators
	\item Different Clock Synchronization
\end{itemize}

EVG/EVR Timing System은 크게 VME 또는 cPCI interface 상에서 동작하며, RAON에서는 VME interface 상에서 운영되는 플랫폼 구조를 채택하였다. Timing System을 구성하는 항목은 크게 아래와 같다.

\begin{itemize}
	\item Hardware 구성 항목
	\begin{itemize}
		\item GPS 수신 장치
		\item SRS Clock source
		\item EVG/EVR/Fanout 보드
		\item MVME 6100/MVME 3100 보드 
		\item VME form factor		
	\end{itemize}
	\item Software 구성항목
	\begin{itemize}
		\item Workbench 3.3, vxWorks 개발 툴
		\item vxWorks 6.9 Realtime Operating System
		\item RTEMS Realtime Operating System
		\item MRFIOC2  
		\item EPICS		
	\end{itemize}
\end{itemize}

상위에 언급된 Hardware 구성은 BNC 및 SMA connector등으로 연결된다. Hardware 구성에 대한 내용은 다음 장에서 상세하게 다룬다. 소프트웨어 구성은 EVG/EVR에 대한 Event code 설정 등에 대한 interface 및 제어를 위하여 MRF2IOC가 개발되어 사용된다. 이는 실시간 환경구성을 위하여 MVME6100 및 MVME3100 보드 상에서 vxWorks 또는 RTEMS라는 실시간 운영체제 상에서 구동 된다.

\chapter{Hardware 구성}

\clearpage

\section{전체 구성도}
\begin{figure}[h!]
	\centering
	\includegraphics[width=1.1\textwidth, height=1.2\textwidth]{./images/timing_conf.eps}
	\caption{Timing System 전체 구성 흐름도}
	\label{fig:timing_conf} 
\end{figure}
 그림 \ref{fig:timing_conf}에서와 같이 RAON 가속기에서 사용될 Timing System에 대한 전체적인 구성 흐름을 볼 수 있다.
 
\clearpage

\section{EVG, Event Generator}

\section{EVR, Event Receiver}

\section{VME System}

\section{Hardware Connection}

\clearpage

\chapter{Software 구성}
\section{개발 서버 구성}
\section{운영체제 구성}

\section{MRF2IOC, EPICS IOC}

\section{VME System}


\begin{itemize}
	\item 
	\item 
\end{itemize}



\clearpage
\bibliographystyle{unsrtnat}
\bibliography{./refs}

\end{document}

