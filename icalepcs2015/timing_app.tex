\documentclass[a4paper,
               %boxit,
               %titlepage,   % separate title page
               %refpage      % separate references
              ]{jacow}

\makeatletter%                           % test for XeTeX where the sequence is by default eps-> pdf, jpg, png, pdf, ...
\ifboolexpr{bool{xetex}}                 % and the JACoW template provides JACpic2v3.eps and JACpic2v3.jpg which might generates errors
 {\renewcommand{\Gin@extensions}{.pdf,%
                    .png,.jpg,.bmp,.pict,.tif,.psd,.mac,.sga,.tga,.gif,%
                    .eps,.ps,%
                    }}{}
\makeatother

\ifboolexpr{bool{xetex} or bool{luatex}} % test for XeTeX/LuaTeX
 {}                                      % input encoding is utf8 by default
 {\usepackage[utf8]{inputenc}}           % switch to utf8

\usepackage[USenglish]{babel}

\ifboolexpr{bool{jacowbiblatex}}%        % if BibLaTeX is used
 {%
  \addbibresource{jacow-test.bib}
  \addbibresource{biblatex-examples.bib}
 }{}

\newcommand\SEC[1]{\textbf{\uppercase{#1}}}

%%
%%   Lengths for the spaces in the title
%%   \setlength\titleblockstartskip{..}  %before title, default 3pt
%%   \setlength\titleblockmiddleskip{..} %between title + author, default 1em
%%   \setlength\titleblockendskip{..}    %afterauthor, default 1em

%\copyrightspace %default 1cm. arbitrary size with e.g. \copyrightspace[2cm]

% testing to fill the copyright space
%\usepackage{eso-pic}
%\AddToShipoutPictureFG*{\AtTextLowerLeft{\textcolor{red}{COPYRIGHTSPACE}}}

\begin{document}

\title{Application using timing system of RAON accelerator\thanks{Work supported by ...}}

\author{S. Lee\thanks{silee7103@ibs.re.kr}, CW. Son\thanks{scwook@ibs.re.kr}, HJ. Jang\thanks{lkcom@ibs.re.kr}, IBS, Daejeon, South Korea\\
       }

\maketitle

%
\begin{abstract}
   RAON is a particle accelerator to research the interaction between the nucleus forming a rare isotope as Korean heavy-ion accelerator. RAON accelerator consists of a number of facilities and equipments as a large-scaled experimental device operating under the distributed environment. For synchronization control between these experimental devices, timing system of the RAON uses the VME-based EVG/EVR system. This paper is intended to test high-speed device control with timing event signal. To test the high-speed performance of the control logic with the minimized event signal delay, we are planing to establish the step motor controller testbed applying the FPGA chip. The testbed controller will be configured with Zynq 7000 series of Xilinx FPGA chip. Zynq as SoC (System on Chip) is divided into PS (Processing System) with PL (Programmable Logic). PS with the dual-core ARM cpu is performing the high-level control logic at run-time on linux operating system. PL with the low-level FPGA I/O signal interfaces with the step motor controller directly with the event signal received from timing system.
   
   This paper describes the content and performance evaluation obtaining from the step motor control through the various synchronized event signal received from the timing system.
\end{abstract}


\section{Introduction}
The RAON\cite{TSHOO:NIMB} is a new heavy ion accelerator under construction in South Korea, which is to produce a variety of stable ion and rare isotope beams to support various researches for the basic science and applied research applications. To produce the isotopes to fulfill the requirements we have planed the several modes of operation scheme which require fine-tuned synchronous controls, asynchronous controls, or both among the accelerator complexes. RAON, which is a large experimental facility, consists of many experimental devices and additional facilities. For synchronized control under the distributed environment, timing system of the RAON uses the VME-based EVG/EVR system.

\section{Timing System}
The timing system uses the Event Generator(EVG)/Event Receiver (EVR) system of the Micro-Research Finland Oy company and VME form factor. The timing system\cite{mrf} consists of an EVG which converts timing events and signals to an optical signal distributed through Fan-Out units to an array of EVRs. The EVRs decode the optical signal and produce hardware and software output signals based on the timing events. The VME controller installed in VME crate uses MVME6100 or 3100 of Emerson company. The VME controllers use real-time operating system, VxWorks 6.9 for a fast response. Those controllers control FPGA-based EVG/EVR boards using mfrioc2 of EPICS IOCs on VxWorks.
\hfil\break
The characteristics of MRF timing system are :
\begin{Itemize}
	\item Event driven system, 256 event codes
	\item Event generation using external RF reference clock
	\item 50 $\sim$  125MHz event clock rate
	\item Events generated
	\begin{Itemize}
		\item From external HW inputs
		\item Two sequencers (up to 2048 events/sequencer)
		\item Multi counters
	\end{Itemize}
	\item Cascaded Event Generators
	\item Different Clock Synchronization
\end{Itemize}

\subsection{Hardware Configuration}

Hardware of the timing module consist of:

\begin{Itemize}
	\item XLI GPS Time System
	\item Rubidium Frequency Standard Clock Source (FS725)
	\item Event Trigger System \\
	(Event Generator/Receiver,Fanout/Concentrator)
	\item MVME 6100/MVME 3100 controller board 
	\item VME Crate (Wiener)
	\item SMA 100a RF Signal Generator
\end{Itemize}
\hfil\break
Fig.~\ref{timing:rack} shows hardware configuration of the test timing system. 
\begin{figure}[!htb]
	\centering
	\includegraphics*[width=65mm]{timing_rack}
	\caption{Timing Hardware configuration }
	\label{timing:rack}
\end{figure}

Referring to the configuration of Fig.~\ref{timing:rack}, GPS receiver synchronizes with rubidium clock and EVG to 1PPS. Signal generator generates RAON reference clock(81.25 MHz, RF master oscillator) to EVG. The two signals of EVG are synchronized by locking the phase. The synchronized signal of EVG is generated and distributed to EVR according to event code of MRFIOC on VME system.

\subsection{Software Configuration}
Software to operate the timing system consist of:
\begin{Itemize}
	\item Workbench 3.3, vxWorks IDE
	\item vxWorks 6.9 Realtime Operating System
	\item EPICS framework (R3.14.12.5)
	\item MRFIOC2/SRSIOC 
	\item Network Time Protocol (NTP)
\end{Itemize}

\section{Application : ZYNQ  }
%본 문서에는 Xilinx의 FPGA Chip 중 Zynq 사용을 위한 내용을 기술한다. Zynq의 특징은 앞서 설명한 FPGA의 특징과 ARM CPU를 내장한 embedded 특징 모두 가지는 하나의 SoC(System on Chip) 이다.  따라서 Zynq Chip에는 기존 FPGA에 가지는 장점에 embedded 형태의 운영체제를 가질 수 있는 장점이 있다. 또한 이 둘간의 interface는 AXI Bus interface를 통한 메모리 맵을 통하여 데이터 통신이 가능하다. 따라서 고속 신호처리를 위하여는 FPGA I/O를 low level(hardware) 단에서 처리를 하며, 상위 레벨과의 인터페이스 및 주요 제어로직은 CPU 모듈에서 처리 할 수 있는 설계가 가능하다. 이는 가속기 제어 operation sequence에 따른 timing event의 external trigger를 입력으로 받아 고속의 신호를 처리하여야 하는 application에 활용 할 수 있다. 또한 MPS(Machine Protection System)와 같은 고속의 application에 적용할 수 있으리라 판단된다.Zynq는 ARM 계열의 CPU를 가지고 있으며 여기에 linux 운영체제를 쉽게 올릴 수 있는 petalinux softeware tool을 제공한다. 

\section{Stepper Motor Testbed}

\section{EPICS Interface}

\begin{figure*}[!tbh]
    \centering
    \includegraphics*[width=\textwidth]{JACpic2v3}

    \caption{Example of a full-width figure showing the JACoW Team at their annual
             meeting in 2012. This figure has a multi-line caption that has to
             be justified rather than centered.}
    \label{l2ea4-f2}
\end{figure*}

\subsection{Title and Author List}

The title should use \SI{14}{pt} bold uppercase letters and be centered on the page.
Individual letters may be lowercase to avoid misinterpretation (e.g., mW, MW).
To include a funding support statement, put an asterisk after the title and
the support text at the bottom of the first column on page~1---in Word,
use a text box; in \LaTeX, use $\backslash$\texttt{thanks}.

The names of authors, their organizations/affiliations and mailing addresses
should be grouped by affiliation and listed in \SI{12}{pt} upper- and lowercase letters.
The name of the submitting or primary author should be first, followed by
the co-authors, alphabetically by affiliation.


\subsection{Section Headings}

Section headings should not be numbered. They should
use  \SI{12}{pt}  bold  uppercase  letters  and  be  centered  in  the
column. All section headings should appear directly above
the text---there should never be a column break between a heading and the
following paragraph.

\subsection{Subsection Headings}

Subsection  headings  should  not  be  numbered.
They should use \SI{12}{pt} italic letters and be left aligned in the column.
Subsection headings use \emph{T}itle \emph{C}ase (or \emph{I}nitial \emph{C}aps)
and should appear directly above the text---there should never be a column break
between a heading and the following paragraph.

\subsubsection{Third-level Headings} are created with the \LaTeX\ command \verb|\subsubsection|.
In the Word templates authors must bold the text themselves; this
heading should be used sparingly. See Table~\ref{style-tab} for its
style details.

\subsection{Paragraph Text}

Paragraphs should use \SI{10}{pt} font and be justified (touch each side) in
the column. The beginning of each paragraph should be indented
approximately \SI{3}{mm} (\SI{0.13}{in}). The last line of a paragraph should not be
printed by itself at the beginning of a column nor should the first line of
a paragraph be printed by itself at the end of a column.

\subsection{Figures, Tables and Equations}

Place figures and tables as close to their place of mention as
possible. Lettering in figures and tables should be large enough to
reproduce clearly. Use of non-approved fonts in figures can lead to
problems when the files are processed. \LaTeX\ users should be sure to use
non-bitmapped versions of Computer Modern fonts in equations (Type\,1 PostScript
or OpenType fonts are required, Their use is described in the JACoW help
pages~\cite{jacow-help}).

Each figure and table must be numbered in ascending order (1, 2, 3, etc.) throughout
the paper. Figure captions (\SI{10}{pt} font) are placed below figures, and table captions are placed above tables. Single-line captions are centered in the column, while captions that span more than one line should be justified. The \LaTeX\ template uses the ‘booktabs’ package to
format tables.

A simple way to introduce figures into a Word document is to place them inside a table that has no borders. This is done in Word as described below.

\textit{Note: If the figure spans both columns, do all steps. If
the figure is contained in a single column, start at step 5.}

\begin{Enumerate}
\item	Insert a continuous section break.
\item	Insert two empty lines (makes later editing easier).
\item	Insert another continuous section break.
\item	Click between the two section breaks and Page Layout $\rightarrow$
        Columns $\rightarrow$ Single.
\item	Insert $\rightarrow$ Table select a one-column, two-row table.
\item	Paste the figure in the first row of the table and adjust the size as appropriate.
\item	Paste/Type the caption in the second row and apply the appropriate figure caption style.
\item	Table $\rightarrow$ Table properties $\rightarrow$ Borders and
        Shading $\rightarrow$ None.
\item	Table $\rightarrow$ Table properties $\rightarrow$ Alignment $\rightarrow$ Center.
\item	Table $\rightarrow$ Table properties $\rightarrow$ Text wrapping $\rightarrow$ None.
\item	Remove blank lines from in and around the table.
\item	If necessary play with the cell spacing and other parameters to improve appearance.
\end{Enumerate}

If a displayed equation needs a number (i.\,e., it will be referenced), place it flush with the right margin of the column (see Eq.~\ref{eq:units}). The equation itself should be indented (centered, if possible). Units should be written using the roman (standard) font,
not the italic font:

\begin{equation}\label{eq:units}
    C_B=\frac{q^3}{3\epsilon_{0} mc}=\SI{3.54}{\micro eV/T}
\end{equation}

\subsection{References}

All bibliographical and web references should be numbered and listed at the
end of the paper in a section called \SEC{References}. When citing a
reference in the text, place the corresponding reference number in square
brackets~\cite{exampl-ref}. The reference citations in the text should be numbered
in ascending order.

A URL may be included as part of a reference, but
its hyperlink should NOT be added. Multiple citations should appear in
the same square bracket~\cite{jacow-help, exampl-ref2, exampl-ref3} or
with ranges, e.g., \cite{jacow-help}--\cite{exampl-ref3} or \cite{jacow-help, exampl-ref, exampl-ref2, exampl-ref3, exampl-last}.

Examples of correctly formatted references can be found at the JACoW website (http://JACoW.org). Once there, click on the ‘for Authors’ link at the top and then on the ‘Formatting
Citations’ link in the left-hand column \cite{jacow-help}.

\subsection{Footnotes}

Footnotes on the title and author lines may be used for acknowledgments,
affiliations and e-mail addresses. A non-numeric sequence of characters (*, \#,
\dag, \ddag) should be used.
Word users---DO NOT use Word's footnote feature (\textbf{Insert}, \textbf{Footnote})
to insert footnotes, as this will create formatting problems. Instead, insert
the title or author footnotes manually in a text box at the bottom of the first column with a
line at the top of the text box to separate the footnotes from the rest of
the paper's text.  The easiest way to do this is to copy the text box from
the JACoW template and paste it into your own document.
These “pseudo footnotes” in the text box should only
appear at the bottom of the first column on the first page.

Any other footnote in the body of the paper should use the normal numeric
sequencing and appear as footnote\footnote{This text should appear
in the column where it was referenced.} in the same column where they are used.

\subsection{Acronyms}

Acronyms should be defined the first time they appear.

\section{styles}

Table~\ref{style-tab} summarizes the fonts and spacings used in the styles of
a JACoW template (these are implemented in the \LaTeX\ class file).
\begin{table}[h!t]
    \setlength\tabcolsep{3.8pt}
    \caption{Summary of Styles}
    \label{style-tab}
    \begin{tabular}{@{}llcc@{}}
        \toprule
        \textbf{Style} & \textbf{Font}               & \textbf{Space}  & \textbf{Space} \\
                       &                             & \textbf{Before} & \textbf{After} \\
        \midrule
                       & \SI{14}{pt}                 & \SI{0}{pt}      & \SI{3}{pt}  \\
          Paper Title  & Upper case except for       &                 &      \\
                       & required lower case letters &                 &      \\   %corrected 080515 vrws requred
                       & Bold                        &                 &      \\
         \midrule
          Author list  & \SI{12}{pt}                 & \SI{9}{pt}      & \SI{12}{pt} \\
                       & Upper and Lower case        &                 &      \\
         \midrule
         Section       & \SI{12}{pt}                 & \SI{9}{pt}      & \SI{3}{pt}  \\
         Heading       & Uppercase                   &                 &      \\
                       & bold                        &                 &      \\
        \midrule
         Subsection    & \SI{12}{pt}                 & \SI{6}{pt}      & \SI{3}{pt}  \\
         Heading       & Initial Caps                &                 &      \\
                       & Italic                      &                 &      \\
        \midrule
         Third-level   & \SI{10}{pt}                 & \SI{6}{pt}           & \SI{0}{pt}  \\
         Heading       & Initial Caps                &                 &      \\
                       & Bold                        &                 &      \\
        \midrule
         Figure        & \SI{10}{pt}                 & \SI{3}{pt}           & \SI{6}{pt}  \\
         Captions      &                             &                 &      \\
        \midrule
         Table         & \SI{10}{pt}                      & \SI{3}{pt}           & \SI{3}{pt}  \\
         Captions      &                             &                 &      \\
        \midrule
         Equations     & \SI{10}{pt} base font            & \SI{12}{pt}          & \SI{12}{pt} \\
        \midrule
         References    & \SI{9/10}{pt}, justified with  \SI{0.25}{in} &      &  \\
                       & hanging indent, reference   & $\ge$\SI{0}{pt} & $\ge$\SI{0}{pt}  \\
                       & number right aligned     &                 &        \\
        \bottomrule
    \end{tabular}
\end{table}

\section{page numbers}

\textbf{DO NOT include any page numbers}. They will be added
when the final proceedings are produced.

\section{templates}

Templates and examples can be retrieved through web
browsers such as Firefox, Chrome and Internet Explorer by saving to disk.

Template documents for the recommended word processing software are
available from the JACoW website (\url{http://JACoW.org}) and exist for
\LaTeX, Microsoft Word (Mac and PC) and OpenOffice for US letter and A4 paper sizes.

Use the correct template for your paper size and version of Word.
Do not transport Microsoft Word documents across platforms, e.g.,
Mac~$\leftrightarrow$~PC. When saving a Word 2010 file (PC), be sure
to click `Embed fonts' in the Save options. Fonts are embedded by default
when printing to PDF on Mac OSX.

Please see the information and help files for authors on the JACoW.org website
for instructions  on  how to install templates in your Microsoft templates folder.

\section{checklist for electronic publication}

\begin{Itemize}
    \item  Use only Times or Times New Roman (standard, bold or italic) and Symbol
           fonts for text---\SI{10}{pt} minimum except references, which can be \SI{9}{pt} or \SI{10}{pt}.
    \item  Figures should use Times or Times New Roman (standard, bold or italic) and
           Symbol fonts when possible---\SI{6}{pt} minimum.
    \item  Check that citations to references appear in sequential order and
           that all references are cited~\cite{exampl-last}.
    \item  Check that the PDF file prints correctly.
    \item  Check that there are no page numbers.
    \item  Check that the margins on the printed version are within \SI{\pm1}{mm}
           of the specifications.
    \item  \LaTeX\ users can check their margins by invoking the
           \texttt{boxit} option.
\end{Itemize}

\section{Conclusion}

Any conclusions should be in a separate section directly preceding
the \SEC{Acknowledgment}, \SEC{Appendix}, or \SEC{References} sections, in that
order.

\section{acknowledgment}
This work is supported by the Rare Isotope Science Project funded by Ministry of Science, ICT and Future Planning\SEC{(MSIP)} and National Research Foundation\SEC{(NRF)} of Korea(Project No. 2011-0032011).

%
% this setting when the default (\flushend)
% => "balance two column" shows bad results
%
\iftrue   % balancing with bad results
	\newpage
	\raggedend
\fi

\section{appendix}
Any appendix should be in a separate section directly preceding
the \SEC{References} section. If there is no \SEC{References} section,
this should be the last section of the paper.

\iffalse  % only for "biblatex"
	\newpage
	\printbibliography

% "biblatex" is not used, go the "manual" way
\else

\begin{thebibliography}{9}   % Use for  1-9  references

%\bibitem{accelconf-ref}
%	C. Petit-Jean-Genaz and J. Poole,
%	``JACoW, A service to the Accelerator Community,''
%	EPAC'04, Lucerne, July 2004, THZCH03,  p.~249,
%	\url{http://www.JACoW.org/e04/papers/THZCH03.PDF}
\bibitem{TSHOO:NIMB} K.~Tshoo,{ et. al},``Experimental systems overview of the Rare Isotope Science Project in Korea'',

\bibitem{mrf}
	Micro-Research Finland Oy website:
	%\menu[,]{for Authors,Help,Using \LaTeX}
	\url{http://www.mrf.fi}%
%	last visit 27 March 2014.\newline \mbox{ }
	\hfill\textcolor{red}{\{no hyperlink, no period after URL\}}

\bibitem{exampl-ref}
	A.N. Other,
	``A Very Interesting Paper'',
	EPAC'96, Sitges, June 1996, MOPCH31 (1996),
	\url{http://www.JACoW.org}\newline \mbox{ } \hfill\textcolor{red}{\{no hyperlink, no period after URL\}}

\bibitem{exampl-ref2}
	F.E.~Black et al.,
	\textit{This is a Very Interesting Book},
	(New York: Knopf, 2007), 52.

\bibitem{exampl-ref3}
    G.B.~Smith et al., ``Title of Paper'',
    MOXAP07, \textit{These Proceedings}, IPAC'14, Dresden, Germany (2014).

	\hspace*{-1.1em}\mbox{\vdots}


\end{thebibliography}

\fi

\end{document}
