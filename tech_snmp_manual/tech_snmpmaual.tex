%\documentclass[11pt,a4paper]{article}
\documentclass[11pt
  , a4paper
  , article
  , oneside
%  , twoside
%  , draft
]{memoir}

\usepackage{control}
\usepackage[numbers]{natbib}
\usepackage{tabulary}


\begin{document}
 
\newcommand{\technumber}{
  RAON Control-Document Series\\
  Revision : v0.1,   Release : August. 04. 2015}
\title{\textbf{RAON\_snmp Manual}}

\author{박미정\thanks{mijoy0909@ibs.re.kr} \\

  Rare Isotope Science Project\\
  Institute for Basic Science, Daejeon, South Korea
}
\date{\today}


\renewcommand{\maketitlehooka}{\begin{flushright}\textsf{\technumber}\end{flushright}}
%\renewcommand{\maketitlehookb}{\centering\textsf{\subtitle}}
%\renewcommand{\maketitlehookc}{C}
%\renewcommand{\maketitlehookd}{D}

\maketitle

\begin{abstract}
본 문서는 EPICS와 SNMP를 통합한 RAON에 최적화된 네트워크 장비 모니터링 시스템에 관한 메뉴얼이다.
\end{abstract}

\chapter{Introduction}
RAON\_snmp는 중이온가속기 제어 시스템을 구성하는 이기종 장비들의 최적화와 관리를 위해 네트워크 장비들이 지원하는 인터넷 표준 프로토콜인 간이 망 관리 프로토콜\footnote{* Official Internet Standards Protocol, http://www.rfc-editor.org/search/standards.php, 2002}(Simple Network Management Protoco, SNMP)과 EPICS\footnote{* EPICS, http://www.aps.anl.gov/epics/about.php, 2011}를 통합한 모니터링 시스템이다. 이 시스템은 SNMP를 지원하고 MIB 파일이 존재하는 네트워크 장비에 적용할 수 있다.

\chapter{Development environ. \& Requirements}

RAON\_snmp개발 환경은 다음과 같다.
\begin{itemize}
\item Debian Linux 7 Wheezy
\item EPICS v3.14.12.4
\item Net-SNMP v5.4.3
\end{itemize}

\section{EPICS development environ.}
중이온가속기 제어 시스템의 모든 다른 기종 장비는 EPICS 단일 제어 플랫폼으로 통합되어 개발될 예정이다. 따라서 각 시스템의 원활한 개발 및 개발자 간의 의사소통, 그리고 개발되는 코드들의 단계별 형상관리를 위해  Git\footnote{* http://git-scm.com/}을 이용한 소스코드 버전 관리 시스템을 구축한다. 중이온가속기의 EPICS 개발 환경은 아래와 같이 소프트웨어의 효율적인 재사용을 위한 siteLibs와 IOC를 개발하는 siteApps로 분리된 표준화 및 자동화된 구조를 가진다.

{\scriptsize
\begin{verbatim}
.
├── downloads              
└── R3.14.12.4             
    ├── base               : EPICS core module
    ├── extensions         : EPICS extension software
    ├── setEpicsEnv.sh     : Dynamic Env Setup Script
    ├── siteApps           : RAON EPICS Application
    └── siteLibs           : RAON EPICS Library (reusing code)
\end{verbatim}
}

\section{Requirements}

SNMP 사용에 필요한 Debian distribution packages 리스트는 다음과 같다. 

\begin{itemize}
\item Net-SNMP v5.4.3
\item snmp v5.4.3
\item snmp-mibs-downloader v1.1
\item snmpd v5.4.3
\item libsnmp-dev v5.4.3
\item libsnmp-perl v5.4.3
\item libsnmp15 v5.4.3 (Debian7-wheezy) / libsnmp30 (Debian8-jessie)
\end{itemize}

또한, SNMP를 사용하기 위해서는 관리되어야 할 장비의 객체들을 계층구조로 모아 놓은 MIB (Management Information Base)가 필요하다. 일반적인 장비의 관리정보는 MIB-2(RFC 1213)에 포함돼 있고, 이에 포함되지 않은 장비들은 각 제조업체에서 제공한다. 일반적인 리눅스 시스템에서 MIB는 아래의 경로에 위치한다.

{\scriptsize
\begin{verbatim}
/usr/local/share/snmp/mibs/ 
/usr/share/mibs
\end{verbatim}
}

RAON\_snmp는 siteLibs의 snmpLib에 mibs디렉토리를 생성하여 MIB파일을 관리한다. 따라서 이 경로를 MIB 검색 경로에 추가하기 위해 EPICS IOC의 st.cmd에 아래와 같은 경로 추가 명령어가 필요하다.  
{\scriptsize
\begin{verbatim}
epicsEnvSet("MIBDIRS", "+${RAON_SITELIBS}/snmpLib/mibs")
\end{verbatim}
}

\chapter{Features}
RAON\_snmp는 다음과 같은 특징이 있다. 

\begin{itemize}
\item Supports SNMPv2/v3 (Authentication, Encryption) \\
EPICS Community 내 존재하는 SNMP Support Module들은 SNMPv2c에 최적화되어 있으며, SNMPv3를 대부분 지원하지 않는다. RAON\_snmp는 SNMPv2c/v3를 지원하며, SNMPv3에서는 강력한 보안을 위해 인증과 암호화를 지원한다. 이에 모니터링 시 응답속도가 빠른 SNMPv2 사용을, 제어 시에는 보안이 강력한 SNMPv3를 사용하는 것을 권장한다.

\item New Record Types \\
RAON\_snmp는 기존의 EPICS 레코드 타입이 아닌 새로운 snmp, snmpstrRecord를 지원한다. snmpRecord는 현재 integer, float, gauge의 데이터 타입의 Read를 지원하며, Write는 integer 타입만 지원한다. 나머지 데이터 타입에 대한 Read/Write도 추후 지원 할 예정이다. snmpstrRecord는 string과 BITS타입의 데이터의 Read를 지원하며, 현재 Write는 지원하지 않는다.

\item New fields \\
앞서 설명한 새로운 레코드에는 SNMP 관련 정보가 필드화 되어있으며, 이는 Chap.4에서 설명한다. 이에 레코드별로 VERS 필드를 사용해 SNMPv2c/v3의 버전을 선택적으로 사용할 수 있다. 
\end{itemize}

\chapter{Record specification}
RAON\_snmp는 기존의 EPICS 레코드 대신 SNMP의 데이터 타입에 따른 새로운 두 가지 레코드를 사용한다.
각 레코드에 대한 정보는 다음과 같다.

\section{snmpRecord}
snmpRecord는 SNMP 명령어 사용을 위해 기존의 필드 외 SNMP 명령어 사용에 필요한 필드를 사용한다. 이는 표 \ref{table:snmprecord}과 같으며 다음과 같은 목적을 가진다. HOST는 장비(Agent)의 주소로 IP 주소나 노드의 이름에 관한 필드이다. COMM은 v2c사용 시 Community String이며, v3 사용 시에 Username을 설정하는 필드이다. OID 필드는 읽거나 쓰고 싶은 장비의 정보 객체를 OID표기법이나 이름으로 설정한다. AUTH, PRIV는 v3 사용 시 암호화와 인증 비밀번호를 설정하는 필드이다. VERS는 레코드가 사용할 SNMP의 버전 설정 필드로 SNMPv2c 사용 시 SNMP\_VERSION\_2c, SNMPv3 사용 시에는 SNMP\_VERSION\_3로 설정한다. 
\clearpage

\begin{table}[!h]
\begin{center}
\begin{tabular}{cccc}\hline
Field & Type & Description & Remark\\ \hline
HOST & String & Host Address \\ \hline
COMM & String & Community string (v2c) / Username (v3) \\ \hline
OIDS & String & OID Name \\ \hline
AUTH & String & Auth Key for strong authentication \\ \hline
PRIV & String & Priv Key for data encryption\\ \hline
VERS & String & SNMP Version & SNMPv2c/v3	 \\ \hline
OVAL & Double & Old Value \\ \hline
VAL & Double & Value \\ \hline
\end{tabular}
\caption{fields of snmpRecord}
\label{table:snmprecord} 
\end{center}
\end{table} 

APC사 Power Distribution Unit의 아웃렛 모니터링과 제어를 위한 snmp레코드 사용의 예는 다음과 같다.

{\scriptsize
\begin{verbatim}
record(snmp, "${A}:${P}_Outlet8_R") {
    field(DESC, "PDU outlet8 control")
    field(DTYP, "SNMP Read")
    field(SCAN, "5 second")
    field(VERS, "${V2C}")
    field(HOST, "${HOST}")
    field(COMM, "${COM}")
    field(OIDS, "${PO}sPDUOutletCtl.8")
}
record(snmp, "${A}:${P}_Outlet8_W") {
    field(DESC, "PDU outlet8 control")
    field(DTYP, "SNMP Write")
    field(SCAN, "5 second")
    field(VERS, "${V3}")
    field(AUTH, "${AUTH_P}")
    field(PRIV, "${PRIV_P}")
    field(HOST, "${HOST}")
    field(COMM, "${USER}")
    field(OIDS, "${PO}sPDUOutletCtl.8")
}
\end{verbatim}
}

\section{snmpstrRecord}
snmpstrRecord는 Sting, BITS 타입의 데이터를 처리하기 때문에 알람 파라미터를 사용하지 않으며, 장비의 정보를 읽는 것만 가능하도록 생성되었다. 추후 Write를 지원할 예정이다. 필드에 대한 정보는 표 \ref{table:snmpstrrecord}와 같으며, OVAL, VAL 필드를 제외하고는 snmpRecord와 같다. 
\clearpage

\begin{table}[!h]
\begin{center}
\begin{tabular}{cccc}\hline
Field & Type & Description & Remark\\ \hline
HOST & String & Host Address \\ \hline
COMM & String & Community string (v2c) / Username (v3) \\ \hline
OIDS & String & OID Name \\ \hline
AUTH & String & Auth Key for strong authentication \\ \hline
PRIV & String & Priv Key for data encryption\\ \hline
VERS & String & SNMP Version & SNMPv2c	 \\ \hline
VAL & String & Value \\ \hline
\end{tabular}
\caption{fields of snmpstrRecord}
\label{table:snmpstrrecord} 
\end{center}
\end{table} 

APC PDU Outlet의 이름을 모니터링하는 snmpstr레코드의 사용 예이다. 

{\scriptsize
\begin{verbatim}
record(snmpstr, "${A}:${P}_Outlet8_Name") {
    field(DESC, "PDU outlet8 Name")
    field(DTYP, "SNMP Read")
    field(SCAN, "5 second")
    field(VERS, "${V2C}")
    field(HOST, "${HOST}")
    field(COMM, "${COM}")
    field(OIDS, "${PO}sPDUOutletCtlName.8")
}
\end{verbatim}
}


\chapter{Issue}
\begin{enumerate}[1)]
\item {scan rate} \par
하나의 DB에서 두 버전의 SNMP를 사용할 때, SNMP 버전별 응답시간의 차이로 인해 IOC 내에서 아래와 같이 dbScan warning 메세지를 받을 수 있다. 현재 scan rate를 5초로 권장하며, 이 부분은 보완할 예정이다.


{\scriptsize
\begin{verbatim}
dbScan warning from '2 second' scan thread:
	Scan processing averages 2.27 seconds (2.22 .. 2.37).
	Over-runs have now happened 10 times in a row.
	To fix this, move some records to a slower scan rate.

dbScan warning from '2 second' scan thread:
	Scan processing averages 2.26 seconds (2.20 .. 2.37).
	Over-runs have now happened 17 times in a row.
	To fix this, move some records to a slower scan rate.

dbScan warning from '2 second' scan thread:
	Scan processing averages 2.25 seconds (2.17 .. 2.37).
	Over-runs have now happened 30 times in a row.
	To fix this, move some records to a slower scan rate.

dbScan warning from '2 second' scan thread:
	Scan processing averages 2.28 seconds (2.17 .. 2.42).
	Over-runs have now happened 55 times in a row.
	To fix this, move some records to a slower scan rate.
\end{verbatim}
}


\end{enumerate}

\clearpage



\end{document}
