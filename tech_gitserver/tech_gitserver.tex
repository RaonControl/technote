%\documentclass[11pt,a4paper]{article}
\documentclass[11pt
  , a4paper
  , article
  , oneside
%  , twoside
%  , draft
]{memoir}

\usepackage{control}
\usepackage[numbers]{natbib}
\usepackage{tabulary}


\begin{document}
 
\newcommand{\technumber}{
  RAON Control-Document Series\\
  Revision : v0.1,   Release : December. 01. 2015}
\title{\textbf{Git Server 구축}}

\author{박미정\thanks{mijoy0909@ibs.re.kr} \\
  Rare Isotope Science Project\\
  Institute for Basic Science, Daejeon, South Korea
}
\date{\today}


\renewcommand{\maketitlehooka}{\begin{flushright}\textsf{\technumber}\end{flushright}}
%\renewcommand{\maketitlehookb}{\centering\textsf{\subtitle}}
%\renewcommand{\maketitlehookc}{C}
%\renewcommand{\maketitlehookd}{D}

\maketitle

\begin{abstract}
본 문서는 Debian Linux 환경에서 Git Server 구축에 관한 메뉴얼이다. 
\end{abstract}


\chapter{Git Server Repository 생성}

\begin{enumerate}
\item git 서버로 사용할 pc로 접속

\scriptsize
{
\begin{verbatim}
mijoy0909@mjpark:~$ ssh ctrluser@10.1.5.88
Host key fingerprint is 9f:34:3d:cc:99:12:85:ff:44:02:15:d3:e5:b6:a8:f0
+--[ECDSA  256]---+
|          .+=o ..|
|          ....o. |
|          .. o  o|
|           =.oo..|
|        S.+ B+ . |
|         oo+...  |
|          oE     |
|                 |
|                 |
+-----------------+

ctrluser@10.1.5.88's password: 
\end{verbatim}
}

\item Bare 저장소 생성\\
git init명령에서 --bare를 사용해 워킹 디렉토리가 없는 저장소인 bare 저장소를 생성한다\footnote{git 서버 설정, "https://git-scm.com/book/ko/v2/Git-서버-서버-설정하기"}. bare 저장소는 관례에 따라  .git으로 끝난다. --shared 옵션은 git이 자동으로 그룹 쓰기 권한을 추가하게 한다.  - 업무기간 
- 연수인력 13.09~15.08 (2년)

\scriptsize
{
\begin{verbatim}
ctrluser@ctrlsv1:~$ cd /opt/
trluser@ctrlsv1:/opt$ mkdir git
ctrluser@ctrlsv1:/opt$ sudo chmod 777 git/
ctrluser@ctrlsv1:/opt$ su ctrlgit
Password: 
ctrlgit@ctrlsv1:/opt$ 
ctrlgit@ctrlsv1:/opt$ cd git/
ctrlgit@ctrlsv1:/opt/git$ mkdir RaonControl.git
ctrlgit@ctrlsv1:/opt/git$ cd RaonControl.git/
ctrlgit@ctrlsv1:/opt/git/RaonControl.git$ git init --bare --shared
Initialized empty shared Git repository in /opt/git/RaonControl.git/
ctrluser@ctrlsv1:/opt/git/RaonControl.git$ cd
\end{verbatim}
}

저장소를 생성하면 다음과 같이 디렉토리에 파일들이 생성 됨을 알 수 있다.

\scriptsize
{
\begin{verbatim}
ctrluser@ctrlsv1:/home/ctrlgit/git_server$ ls -al
total 40
drwxrwsr-x 7 ctrluser ctrluser 4096 Nov 17 17:48 .
drwxrwxrwx 3 root     root     4096 Nov 17 17:47 ..
drwxrwsr-x 2 ctrluser ctrluser 4096 Nov 17 17:48 branches
-rwxrw-r-- 1 ctrluser ctrluser  126 Nov 17 17:48 config
-rw-rw-r-- 1 ctrluser ctrluser   73 Nov 17 17:48 description
-rw-rw-r-- 1 ctrluser ctrluser   23 Nov 17 17:48 HEAD
drwxrwsr-x 2 ctrluser ctrluser 4096 Nov 17 17:48 hooks
drwxrwsr-x 2 ctrluser ctrluser 4096 Nov 17 17:48 info
drwxrwsr-x 4 ctrluser ctrluser 4096 Nov 17 17:48 objects
drwxrwsr-x 4 ctrluser ctrluser 4096 Nov 17 17:48 refs
\end{verbatim}
}
\end{enumerate}


\chapter{Git Server 계정 생성}
git 저장소의 권한 관리 및 사용을 위해 git계정을 생성한다. 이 git계정은 서버 접속을 위한 수단일 뿐 사용자의 데이터 commit에는 아무런 영향을 주지 않는다.

\scriptsize
{
\begin{verbatim}
ctrluser@ctrlsv1:~$ sudo adduser ctrlgit
Adding user `ctrlgit' ...
Adding new group `ctrlgit' (1001) ...
Adding new user `ctrlgit' (1001) with group `ctrlgit' ...
Creating home directory `/home/ctrlgit' ...
Copying files from `/etc/skel' ...
Enter new UNIX password: 
Retype new UNIX password: 
passwd: password updated successfully
Changing the user information for ctrlgit
Enter the new value, or press ENTER for the default
Full Name []: ctrluser@ctrlsv1:/opt/git/RaonControl.git$ cd
ctrluser@ctrlsv1:~$ sudo adduser ctrlgit
Adding user `ctrlgit' ...
Adding new group `ctrlgit' (1001) ...
Adding new user `ctrlgit' (1001) with group `ctrlgit' ...
Creating home directory `/home/ctrlgit' ...
Copying files from `/etc/skel' ...
Enter new UNIX password: 
Retype new UNIX password: 
passwd: password updated successfully
Changing the user information for ctrlgit
Enter the new value, or press ENTER for the default
Full Name []: MiJeong Park
Room Number []: 
Work Phone []: 
Home Phone []: 
Other []: mijoy0909@ibs.re.kr
Is the information correct? [Y/n] y
ctrluser@ctrlsv1:~$ 
\end{verbatim}
}

\chapter{SSH 접속}
앞서 생성된 Ctrlgit계정에 쓰기 권한이 필요한 사용자의 SSH 공개키를 서버의 $\sim$/.ssh/authorized\_keys 파일에 등록하기 위한 과정이다.

\section{Git Client SSH 설정}
먼저 $\sim$/.ssh 디렉토리를 확인해 공개키가 있는지 확인한다. $\sim$/.ssh 디렉토리가 존재하지 않다면 공개키는 존재하지 않으며 $\sim$/ 디렉토리에 .ssh 디렉토리를 생성해야한다. 공개키는 id\_dsa나 id\_rsa의 파일명으로 존재하며 .pub 확장자가 붙은 파일이 공개키이며, 나머지는 개인키이다. 공개키가 존재하지 않다면 아래의 과정을 따른다.

\scriptsize
{
\begin{verbatim}
mijoy0909@mjpark:~/.ssh$ 
mijoy0909@mjpark:~/.ssh$ ssh-keygen
Generating public/private rsa key pair.
Enter file in which to save the key (/home/mijoy0909/.ssh/id_rsa): press the Enter key
Enter passphrase (empty for no passphrase):
Enter same passphrase again: 
Your identification has been saved in /home/mijoy0909/.ssh/id_rsa.
Your public key has been saved in /home/mijoy0909/.ssh/id_rsa.pub.
The key fingerprint is:
65:a5:ba:e9:07:79:b0:52:92:41:a2:3c:37:dd:8e:94 mijoy0909@mjpark
The key's randomart image is:
+--[ RSA 2048]----+
|    ...     .    |
| . . o.o   o     |
|  + o Eo. +      |
|   o oooo+       |
|      .oS+       |
|      . +o.      |
|       .oo       |
|       .  .      |
|        ..       |
+-----------------+
mijoy0909@mjpark:~/.ssh$
mijoy0909@mjpark:~/.ssh$ ls -al
total 26K
drwx------  2 mijoy0909 mijoy0909 4.0K Nov 19 14:09 .
drwxr-xr-x 10 mijoy0909 mijoy0909 4.0K Feb  9  2015 ..
-rw-r--r--  1 mijoy0909 mijoy0909 1.2K Aug 22  2014 config
-rw-r--r--  1 mijoy0909 mijoy0909 1.2K Aug 22  2014 config~
-rw-------  1 mijoy0909 mijoy0909 1.7K Nov 17 17:53 id_rsa
-rw-r--r--  1 mijoy0909 mijoy0909  398 Nov 17 17:53 id_rsa.pub   ****
-rw-r--r--  1 mijoy0909 mijoy0909 8.1K Nov 17 14:38 known_hosts
-rw-r--r--  1 mijoy0909 mijoy0909 3.5K Sep 25  2013 known_hosts~
\end{verbatim}
}

위와 같이 생성된 공개키는 파일인 id\_rsa.pub는 scp 명령어 또는 메일을 통해 git 서버에 등록한다. 

\scriptsize
{
\begin{verbatim}
mijoy0909@mjpark:~/.ssh$ scp id_rsa.pub ctrluser@10.1.5.88:/tmp/id_rsa.mjpark.pub
Host key fingerprint is 9f:34:3d:cc:99:12:85:ff:44:02:15:d3:e5:b6:a8:f0
+--[ECDSA  256]---+
|          .+=o ..|
|          ....o. |
|          .. o  o|
|           =.oo..|
|        S.+ B+ . |
|         oo+...  |
|          oE     |
|                 |
|                 |
+-----------------+

ctrluser@10.1.5.88's password: 
id_rsa.pub                                        100%  398     0.4KB/s   00:00    
mijoy0909@mjpark:~/.ssh$ 
\end{verbatim}
}


\section{Git Server SSH 설정}
먼저 SSH 사용을 위해 .ssh 디렉토리 생성 및 권한을 설정한다.

\scriptsize
{
\begin{verbatim}
ctrluser@ctrlsv1:~$ su ctrlgit 
Password: 
ctrlgit@ctrlsv1:/home/ctrluser$ 
ctrlgit@ctrlsv1:~$ mkdir .ssh && chmod 700 .ssh
ctrlgit@ctrlsv1:~$ touch .ssh/authorized_keys && chmod 600 .ssh/authorized_keys
\end{verbatim}
}

클라이언트의 접근을 허용하기 위해 앞서 클라이언트에게 받은 공개키를 등록한다. 

\scriptsize
{
\begin{verbatim}
ctrlgit@ctrlsv1:~/.ssh$ cat /tmp/id_rsa.mjpark.pub >> ~/.ssh/authorized_keys 
\end{verbatim}
}
\chapter{Git Client 정보, 리모트 저장소 등록 및 테스트}
\section{client 정보 등록}
저장된 사용자 정보는 추후 git commit시 다음과 같이 log에 남는다.

\scriptsize
{
\begin{verbatim}
mijoy0909@mjpark:~/RaonControl$ git config --global user.name "MiJeong Park"
mijoy0909@mjpark:~/RaonControl$ git config --global user.email "mijoy0909@ibs.re.kr"

-------------------------------------------------------------------------------------

mijoy0909@mjpark:~/RaonControl$ git log
commit d0f9a13736b573cf8febfd7f0238a6cd93e1c8d9
Author: MiJeong Park <mijoy0909@ibs.re.kr>
Date:   Thu Nov 19 14:05:52 2015 +0900

rm test.c
\end{verbatim}
}

\section{기존의 디렉토리 및 파일을 git 서버에 저장}
기존에 사용중이던 디렉토리 및 파일을 git 서버로 저장하기 위해서는 다음과 같은 과정을 거친다.

\scriptsize
{
\begin{verbatim}
mijoy0909@mjpark:~$ cd RaonControl/
mijoy0909@mjpark:~/RaonControl$ git init
Initialized empty Git repository in /home/mijoy0909/RaonControl/.git/
mijoy0909@mjpark:~/RaonControl$ git add .
mijoy0909@mjpark:~/RaonControl$ git commit -m "git tset"
[master (root-commit) 12dfe92] git tset
1 file changed, 6 insertions(+)
create mode 100644 testgit.c
mijoy0909@mjpark:~/RaonControl$ 
mijoy0909@mjpark:~/RaonControl$ git remote add origin ctrlgit@10.1.5.88:/opt/git/RaonControl.git
mijoy0909@mjpark:~/RaonControl$ 
mijoy0909@mjpark:~/RaonControl$ git push origin master
Host key fingerprint is 9f:34:3d:cc:99:12:85:ff:44:02:15:d3:e5:b6:a8:f0
+--[ECDSA  256]---+
|          .+=o ..|
|          ....o. |
|          .. o  o|
|           =.oo..|
|        S.+ B+ . |
|         oo+...  |
|          oE     |
|                 |
|                 |
+-----------------+

/usr/bin/xauth:  file /home/ctrlgit/.Xauthority does not exist
Counting objects: 3, done.
Delta compression using up to 8 threads.
Compressing objects: 100% (2/2), done.
Writing objects: 100% (3/3), 286 bytes, done.
Total 3 (delta 0), reused 0 (delta 0)
To ctrlgit@10.1.5.88:/opt/git/RaonControl.git
* [new branch]      master -> master
mijoy0909@mjpark:~/RaonControl$ 
\end{verbatim}
}

\section{git 서버에서 clone}
git clone 방법은 기존의 방법과 동일하다.

\scriptsize
{
\begin{verbatim}
mijoy0909@mjpark:~$ git clone ctrlgit@10.1.5.88:/opt/git/RaonControl.git .
\end{verbatim}
}

\scriptsize
{
\section{git 서버에 push}
\begin{verbatim}
mijoy0909@mjpark:~/RaonControl$ mkdir mj_test
mijoy0909@mjpark:~/RaonControl$ git add mj_test
mijoy0909@mjpark:~/RaonControl$ git commit -m "add directory"
mijoy0909@mjpark:~/RaonControl$ git push origin master
\end{verbatim}
}
\clearpage


\end{document}
