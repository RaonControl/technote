% !TeX spellcheck = en_US
%\documentclass[11pt,a4paper]{article}
\documentclass[11pt
  , a4paper
  , article
  , oneside
%  , twoside
%  , draft
]{memoir}

\usepackage{control}
\usepackage[numbers]{natbib}


\begin{document}

\newcommand{\technumber}{
  RAON Control-Document Series\\
  Revision : v1.0,   Release : 2015-03-16 fixed date}
\title{\textbf{Application using timing system of RAON accelerator}}

\author{이상일\thanks{silee7103@ibs.re.kr}, 손창욱, 이정한 \\

  Rare Isotope Science Project\\
  Institute for Basic Science, Daejeon, South Korea
}
\date{\today}

\renewcommand{\maketitlehooka}{\begin{flushright}\textsf{\technumber}\end{flushright}}
%\renewcommand{\maketitlehookb}{\centering\textsf{\subtitle}}
%\renewcommand{\maketitlehookc}{C}
%\renewcommand{\maketitlehookd}{D}

\maketitle

\begin{abstract}
RAON is a particle accelerator to research the interaction between the nucleus forming a rare isotope as Korean heavy-ion accelerator. RAON accelerator consists of a number of facilities and equipments as a large-scaled experimental device operating under the distributed environment. For synchronization control between these experimental devices, timing system of the RAON uses the VME-based EVG/EVR system. This paper is intended to test high-speed device control with timing event signal. To test the high-speed performance of the control logic with the minimized event signal delay, we are planing to establish the step motor controller testbed applying the FPGA chip. The testbed controller will be configured with Zynq 7000 series of Xilinx FPGA chip. Zynq as SoC (System on Chip) is divided into PS (Processing System) with PL (Programmable Logic). PS with the dual-core ARM cpu is performing the high-level control logic at run-time on linux operating system. PL with the low-level FPGA I/O signal interfaces with the step motor controller directly with the event signal received from timing system.

This paper describes the content and performance evaluation obtaining from the step motor control through the various synchronized event signal received from the timing system.

\end{abstract}

\clearpage
\bibliographystyle{unsrtnat}
\bibliography{./refs}

\end{document}

