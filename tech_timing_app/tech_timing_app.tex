% !TeX spellcheck = en_US
%\documentclass[11pt,a4paper]{article}
\documentclass[11pt
-  , a4paper
  , article
  , oneside
%  , twoside
%  , draft
]{memoir}

\usepackage{geometry}
\geometry{a4paper, margin=0.5in}
\usepackage{control}
\usepackage{multicol}
\usepackage[numbers]{natbib}
\usepackage{booktabs}
\usepackage{verbatim}
\usepackage{framed}
\usepackage{xcolor}


\begin{document}
\newcommand{\technumber}{
  RAON Control-Document Series\\
  Revision : v1.0,   Release : 2015-10-12 fixed date}
\title{\textbf{Application using timing system of RAON accelerator}}

\author{이상일\thanks{silee7103@ibs.re.kr}, 손창욱, 장효재 \\

  Rare Isotope Science Project\\
  Institute for Basic Science, Daejeon, South Korea
}
\date{\today}

\renewcommand{\maketitlehooka}{\begin{flushright}\textsf{\technumber}\end{flushright}}
%\renewcommand{\maketitlehookb}{\centering\textsf{\subtitle}}
%\renewcommand{\maketitlehookc}{C}
%\renewcommand{\maketitlehookd}{D}

\maketitle

\begin{abstract}
RAON is a particle accelerator to research the interaction between the nucleus forming a rare isotope as Korean heavy-ion accelerator. RAON accelerator consists of a number of facilities and equipments as a large-scaled experimental device operating under the distributed environment. For synchronization control between these experimental devices, timing system of the RAON uses the VME-based EVG/EVR system. This paper is intended to test high-speed device control with timing event signal. To test the high-speed performance of the control logic with the minimized event signal delay, we are planing to establish the step motor controller testbed applying the FPGA chip. The testbed controller will be configured with Zynq 7000 series of Xilinx FPGA chip. Zynq as SoC (System on Chip) is divided into PS (Processing System) with PL (Programmable Logic). PS with the dual-core ARM cpu is performing the high-level control logic at run-time on linux operating system. PL with the low-level FPGA I/O signal interfaces with the step motor controller directly with the event signal received from timing system.

This paper describes the content and performance evaluation obtaining from the step motor control through the various synchronized event signal received from the timing system.
\end{abstract}

\begin{multicols}{2}
	
	
\chapter{Introduction}
The RAON\cite{TSHOO:NIMB} is a new heavy ion accelerator under construction in South Korea, which is to produce a variety of stable ion and rare isotope beams to support various researches for the basic science and applied research applications. To produce the isotopes to fulfill the requirements we have planed the several modes of operation scheme which require fine-tuned synchronous controls, asynchronous controls, or both among the accelerator complexes. RAON, which is a large experimental facility, consists of many experimental devices and additional facilities. For synchronized control under the distributed environment, timing system of the RAON uses the VME-based EVG/EVR system. 
이러한 많은 실험 장치들을 제어하기 위한 제어 시스템들은 넓은 범위로 분산되어 구성되어 있으며 이러한 분산 환경에서의 많은 제어시스템들을 전체의 하나의 제어시스템으로 운영하기 위하여 RAON control system은 정밀한 타임 동기화가 필요하다. RAON 제어시스템은 정밀한 타이밍 동기화를 위하여 EVG/EVR 시스템을 사용한다. 

\chapter{Timing System}
타이밍 시스템은 타이밍 모듈과 네트워크로 구성되어 있다. 타이밍 모듈은 MRF 사의 EVG, EVR 제품을 사용하며 폼펙터는 VME이다. 이 모듈들은 VME crate에 설치되며 VME 모듈을 제어하기 위해서 VME controller인 Emerson 사의 MVME6100을 사용한다. MVME6100에는 VxWorks 6.9를 OS로 사용한다.
EPICS의 모듈 중의 하나인 mrfIoc2를 사용하여 EVG와 EVR을 제어한다. 네트워크는 광케이블로 구성한다. 
\section{Event Generator / Receiver}
EVG/EVR Timing System에 대한 특징은 아래와 같다.

\begin{itemize}
	\item Event driven system, 255 event codes
	\item 외부 RF reference clock을 이용한 event 신호 생성
	\item 50 ~ 125MHz Event clock rate
	\item Events generated
	\begin{itemize}
		\item From external HW inputs
		\item Two sequencers (up to 2048 events/sequencer)
		\item Multi counters
	\end{itemize}
	\item Cascaded Event Generators
	\item Different Clock Synchronization
\end{itemize}
EVG/EVR Timing System은 크게 VME 또는 cPCI interface 상에서 동작하며, RAON에서는 VME interface 상에서 운영되는 플랫폼 구조를 채택하였다. Timing System을 구성하는 항목은 크게 아래와 같다.


상위에 언급된 Hardware 구성은 SMA connector로 연결된다. Hardware 구성에 대한 내용은 다음 장에서 상세하게 다룬다. 소프트웨어 구성은 EVG/EVR에 대한 Event code 설정 등에 대한 interface 및 제어를 위하여 MRF2IOC가 개발되어 사용된다. 이는 실시간 환경구성을 위하여 MVME6100 및 MVME3100 보드 상에서 vxWorks 또는 RTEMS라는 실시간 운영체제 상에서 구동 된다.

\section{Hardware}
EVG/EVR Timing System에 대한 특징은 아래와 같다.
\begin{itemize}
	\item GPS 수신 장치
	\item SRS Clock source
	\item EVG/EVR/Fanout 보드
	\item MVME 6100/MVME 3100 보드 
	\item VME form factor		
\end{itemize}
	

\section{Software}
EVG/EVR Timing System에 대한 특징은 아래와 같다.
\begin{itemize}
	\item Workbench 3.3, vxWorks 개발 툴
	\item vxWorks 6.9 Realtime Operating System
	\item RTEMS Realtime Operating System
	\item MRFIOC2  
	\item EPICS		
\end{itemize}

	


\hfil\break

\chapter{Application : ZYNQ  }
본 문서에는 Xilinx의 FPGA Chip 중 Zynq 사용을 위한 내용을 기술한다. Zynq의 특징은 앞서 설명한 FPGA의 특징과 ARM CPU를 내장한 embedded 특징 모두 가지는 하나의 SoC(System on Chip) 이다.  따라서 Zynq Chip에는 기존 FPGA에 가지는 장점에 embedded 형태의 운영체제를 가질 수 있는 장점이 있다. 또한 이 둘간의 interface는 AXI Bus interface를 통한 메모리 맵을 통하여 데이터 통신이 가능하다. 따라서 고속 신호처리를 위하여는 FPGA I/O를 low level(hardware) 단에서 처리를 하며, 상위 레벨과의 인터페이스 및 주요 제어로직은 CPU 모듈에서 처리 할 수 있는 설계가 가능하다. 이는 가속기 제어 operation sequence에 따른 timing event의 external trigger를 입력으로 받아 고속의 신호를 처리하여야 하는 application에 활용 할 수 있다. 또한 MPS(Machine Protection System)와 같은 고속의 application에 적용할 수 있으리라 판단된다.
Zynq는 ARM 계열의 CPU를 가지고 있으며 여기에 linux 운영체제를 쉽게 올릴 수 있는 petalinux softeware tool을 제공한다. 

\chapter{Stepper Motor Testbed}

\chapter{EPICS Interface}

\end{multicols}
	
	



\newpage
\vspace*{0.5cm}
\renewcommand{\bibname}{References}
\begin{thebibliography}{9}   % Use for  1-9  references
\bibitem{TSHOO:NIMB} K.~Tshoo,{ et. al},``Experimental systems overview of the Rare Isotope Science Project in Korea'',
\bibitem{MRFIOC} MRFIOC2 ``http://www.'',
\bibitem{EPICS} EPICS ``http://www.'',
\bibitem{ZYNQ} ZYNQ ``http://www.'',
\bibitem{AnalogDevice} Analog Device ``http://www.'',

\end{thebibliography}



\end{document}

